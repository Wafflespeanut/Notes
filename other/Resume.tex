% Copyright 2006-2015 Xavier Danaux (xdanaux@gmail.com).
%
% This work may be distributed and/or modified under the
% conditions of the LaTeX Project Public License version 1.3c,
% available at http://www.latex-project.org/lppl/.

\documentclass[11pt,a4paper,sans]{moderncv}        % possible options include font size ('10pt', '11pt' and '12pt'), paper size ('a4paper', 'letterpaper', 'a5paper', 'legalpaper', 'executivepaper' and 'landscape') and font family ('sans' and 'roman')

% moderncv themes
\moderncvstyle{banking}                            % style options are 'casual' (default), 'classic', 'banking', 'oldstyle' and 'fancy'
\moderncvcolor{black}                              % color options 'black', 'blue' (default), 'green', 'grey', 'orange', 'purple' and 'red'
% \renewcommand{\familydefault}{\sfdefault}          % to set the default font; use '\sfdefault' for the default sans serif font, '\rmdefault' for the default roman one, or any tex font name
\nopagenumbers                                     % uncomment to suppress automatic page numbering for CVs longer than one page

% character encoding
% \usepackage[utf8]{inputenc}                        % if you are not using xelatex ou lualatex, replace by the encoding you are using
% \usepackage{CJKutf8}                               % if you need to use CJK to typeset your resume in Chinese, Japanese or Korean

% adjust the page margins
\usepackage[scale=0.75]{geometry}
% \setlength{\hintscolumnwidth}{3cm}                 % if you want to change the width of the column with the dates
% \setlength{\makecvtitlenamewidth}{10cm}            % for the 'classic' style, if you want to force the width allocated to your name and avoid line breaks. be careful though, the length is normally calculated to avoid any overlap with your personal info; use this at your own typographical risks...

% personal data
\name{Ravi}{Shankar}
% \title{Resumé}
% \address{street and number}{postcode city}{country} % the "postcode city" and "country" arguments can be omitted or provided empty
% \address{\textbf{UG Final Year (B.E. Aeronautics)}}{Madras Institute of Technology}
\phone[mobile]{+91~9551208590}                     % the optional "type" of the phone can be "mobile" (default), "fixed" or "fax"
% \phone[fixed]{+2~(345)~678~901}
% \phone[fax]{+3~(456)~789~012}
\email{wafflespeanut@gmail.com}
% \homepage{wafflespeanut.github.io}
% \social[linkedin]{wafflespeanut}
% \social[twitter]{wafflespeanut}
\social[github]{wafflespeanut}
% \extrainfo{}
% \photo[64pt][0.4pt]{picture}                       % '64pt' is the height the picture must be resized to, 0.4pt is the thickness of the frame around it (put it to 0pt for no frame) and 'picture' is the name of the picture file
% \quote{Some quote}                                 
% bibliography adjustements (only useful if you make citations in your resume, or print a list of publications using BibTeX)
%   to show numerical labels in the bibliography (default is to show no labels)
\makeatletter\renewcommand*{\bibliographyitemlabel}{\@biblabel{\arabic{enumiv}}}\makeatother
%   to redefine the bibliography heading string ("Publications")
%\renewcommand{\refname}{Articles}

% bibliography with mutiple entries
%\usepackage{multibib}
%\newcites{book,misc}{{Books},{Others}}

\definecolor{linky}{rgb}{0.1, 0.2, 0.9}
\newcommand\chref[3][linky]{\href{#2}{\color{#1}#3}}

%----------------------------------------------------------------------------------
%            content
%----------------------------------------------------------------------------------

\begin{document}
% \begin{CJK*}{UTF8}{gbsn}                         % to typeset your resume in Chinese using CJK

%-----       resume       ---------------------------------------------------------

\vspace*{-1.8\baselineskip}
\makecvtitle
\vspace{-1.75\baselineskip}

\section{Education}
\cventry{Ponjesly Public Matriculation School}{440/500 (91\%)}{10th grade}{2009 -- 2010}{}{}  % arguments 3 to 6 can be left empty
\cventry{DVD Higher Secondary School}{1131/1200 (94.25\%)}{12th grade}{2011 -- 2012}{}{}
\cventry{Madras Institute of Technology}{CGPA: 6.23}{Bachelors Degree - Aeronautics}{2012 -- 2016}{}{}

% \section{Master thesis}
% \cvitem{title}{\emph{Title}}
% \cvitem{supervisors}{Supervisors}
% \cvitem{description}{Short thesis abstract}

\section{Experience}
\cventry{Genome Life Sciences}{Ref. Giriraj Namachivayam (Product Manager)}{Backend Developer Intern}{January, 2016 -- May, 2016}{}
{\begin{itemize}
\item Introduced Rust language to the team, and rewrote a number of Bash and Python scripts in Rust, which showed a drastic improvement in performance.
\item Wrote a few utilities in Rust for bulk parallel processing of chromosome and DNA sequence data in FASTQ, SAM and VCF file formats.
\item Earned the "game changer" award.
\end{itemize}}
\cventry{Genome Life Sciences}{Ref. Giriraj Namachivayam (Product Manager)}{Bioinformatics Programmer}{June, 2016 -- Present}{}
{\begin{itemize}
\item Wrote an utility which collects known species data from various references and tries to predict the species from the given DNA sequence in O(1) time or O(log-n) time depending on the space-time tradeoff.
\item Wrote a few more utilities for analysis of bio-data, some of which now power the backend pipelines.
\end{itemize}}

\section{Programming skills}
\cvitem{Languages}{Python, Rust, HTML5, Javascript, CSS, Bash}
\cvitem{Technologies}{Git, Mercurial, Flask}

\section{Open source contributions}
\begin{itemize}
\item \chref{https://github.com/servo/servo/commits?author=Wafflespeanut}{Active contributor} and \chref{https://blog.servo.org/2016/01/11/twis-47/}{reviewer} for the \chref{https://github.com/servo/servo}{Servo browser engine} project, primarily concentrating on the python code used by the build system, style system, and mentoring the newcomers.
\newline
\textit{Notable contributions:}
\begin{itemize}
\item A \chref{https://github.com/Wafflespeanut/rust-sorty}{compiler plugin} for checking sorted order of declaration statements.
\item Various \chref{https://github.com/servo/highfive/commits?author=Wafflespeanut}{handlers} for \chref{https://github.com/servo/highfive}{highfive} (a bot that responds to Github webhook payloads by welcoming newcomers, assign/tag issues and pull requests, post build failures, etc.) and a "mark and sweep" \chref{https://github.com/servo/highfive/pull/115}{JSON cleaner} for its tests.
\item A \chref{https://github.com/servo-automation/servo-wpt}{watcher} that tests Servo builds in a dedicated machine, analyzes the logs, maintains a database of "rr" recordings of intermittent failures, and uses the Github API to file issues or comments to notify the people who work on such issues.
\end{itemize}
\item Occassional contributor to the \chref{https://github.com/rust-lang/rust}{Rust programming language}, its documentation and related tooling.
\item \chref{https://mozillians.org/en-US/u/wafflespeanut}{Mozillian} since the summer of 2015.
\end{itemize}

\newpage

\section{Personal projects}
\begin{itemize}
\item \textbf{\chref{https://github.com/servo-automation/highfive}{Highfive}}: A complete rework of all the webhook event handlers from Servo's \chref{https://github.com/servo/highfive}{highfive} for efficiency. Apart from handling Github's webhook events, it supports sharing the load between multiple bots, and offers configuration for individual repositories, events and their corresponding handlers.
\item \textbf{\chref{https://github.com/Wafflespeanut/rust-helix}{Helix}}: A Rust library to map short DNA sequence reads to the reference genome. It makes use of suffix array to generate the Burrows-Wheeler transform, from which an FM-index is built and used for finding exact matches in O(1) time.
\item \textbf{\chref{https://github.com/Wafflespeanut/rust-catalog}{Catalog}}: A "file-backed" map written in Rust, for maintaining key/value pairs in a file (sorted with respect to their hashes), which uses binary search and file seeking to "get" the values in O(log-n) time.
\item \textbf{\chref{https://github.com/Wafflespeanut/biographer}{Biographer}}: A command-line based private diary written in Python, which allows users to write their everyday stories, view them, or search through them later. It makes use of a simple shifting cipher to encrypt/decrypt the contents. It also contains a Rust library, which uses FFI and parallelization to reduce the searching time by a factor of $\approx 100$.
\item \textbf{\chref{https://github.com/Wafflespeanut/free-fall}{Free fall}}: A terminal based 2D-ASCII game written in Rust, where the users try to save a jumper from hitting the cliffs. The game makes use of the terminal's raw mode and interacts with the Unix C libraries for polling the keystroke inputs and prints thousands of characters frame by frame to indicate motion.
\end{itemize}

% \section{Languages}
% \cvitemwithcomment{Language 1}{Skill level}{Comment}

% \section{Skills:}
% \cvdoubleitem{category 1}{XXX, YYY, ZZZ}{category 2}{XXX, YYY, ZZZ}

\section{Miscellaneous}
\begin{itemize}
\item I also play with code and make some cool stuff in my free time:
\begin{itemize}
\item A \chref{https://github.com/Wafflespeanut/flight-2016}{responsive website} (for a symposium) without the use of any external libraries.
\item A \chref{https://github.com/Wafflespeanut/volcano-min}{method} for selective-plotting of volcano plots.
\item A \chref{https://github.com/Wafflespeanut/AISH}{CSS injector} that slowly injects a stylesheet into a style element in the DOM and gets rendered in realtime.
\item An \chref{https://ascii-gen.herokuapp.com/}{ASCII Art Generator} for JPEG/PNG images which \chref{http://wafflespeanut.github.io/blog/2017/03/01/ascii-sketch/}{extracts the necessary details} for generating the ASCII sketch.
\end{itemize}
\item Blogger since 2013 on \chref{https://wafflescrazypeanut.wordpress.com/}{wafflescrazypeanut.wordpress.com} and now, at \chref{https://wafflespeanut.github.io/}{wafflespeanut.github.io}
\item \chref{https://physics.stackexchange.com/users/11062}{Contributor and reviewer} of posts at Physics Stack Exchange for two years (2013-2015).
\item I'm also an avid gamer, hobby composer and juggler when I'm AFK.
\end{itemize}

% \section{Extra 1}
% \cvlistitem{Item 1}
% \cvlistitem{Item 2. This item is particularly long and therefore normally spans over several lines. Did you notice the indentation when the line wraps?}

% \section{Extra 2}
% \cvlistdoubleitem{Item 1}{Item 4}
% \cvlistdoubleitem{Item 2}{Item 5\cite{book1}}
% \cvlistdoubleitem{Item 3}{Item 6. Like item 3 in the single column list before, this item is particularly long to wrap over several lines.}

% \section{References}
% \begin{cvcolumns}
%   \cvcolumn{Category 1}{\begin{itemize}\item Person 1\item Person 2\item Person 3\end{itemize}}
%   \cvcolumn{Category 2}{Amongst others:\begin{itemize}\item Person 1, and\item Person 2\end{itemize}(more upon request)}
%   \cvcolumn[0.5]{All the rest \& some more}{\textit{That} person, and \textbf{those} also (all available upon request).}
% \end{cvcolumns}

% Publications from a BibTeX file without multibib
%  for numerical labels: \renewcommand{\bibliographyitemlabel}{\@biblabel{\arabic{enumiv}}}% CONSIDER MERGING WITH PREAMBLE PART
%  to redefine the heading string ("Publications"): \renewcommand{\refname}{Articles}
% \nocite{*}
% \bibliographystyle{plain}
% \bibliography{publications}                      % 'publications' is the name of a BibTeX file

% Publications from a BibTeX file using the multibib package
% \section{Publications}
% \nocitebook{book1,book2}
% \bibliographystylebook{plain}
% \bibliographybook{publications}                  % 'publications' is the name of a BibTeX file
% \nocitemisc{misc1,misc2,misc3}
% \bibliographystylemisc{plain}
% \bibliographymisc{publications}                  % 'publications' is the name of a BibTeX file

\clearpage

%-----       letter       ---------------------------------------------------------

% recipient data
\recipient{EMBL Recruitment team}{European Molecular Biology Laboratory}
\date{March 19, 2017}
\opening{Dear member(s) of the committee,}
% \enclosure[Attached]{curriculum vit\ae{}}        % use an optional argument to use a string other than "Enclosure", or redefine \enclname

\makelettertitle

I'm writing to apply for the Backend Developer position advertised in your website. I'm currently a research programmer at \textit{Genome Life Sciences}, a bioinformatics company, where I've been writing code for backend engines for over an year. I'm extremely interested in obtaining the position at EMBL, where I can continue exploring new things in the domain, improve my skills and apply them in problems that matter in the field.

Even though I dropped out of college in my final year, my academic experience and diversified knowledge in Aeronautics, Physics and Computer Science is what makes me believe that I could be an effective asset in your research environment. I have 3 years of experience in Python and 2 years in Rust (a compiled language that offers the performance of C++).

Throughout the year in my current job, I've been developing applications for handling large datasets in FASTQ, SAM and VCF file formats, to produce results as fast as possible while utilizing the available resources. One such application was FASTQ+, a parallel processing engine for validating FASTQ files. With help from the bioinformaticians in our company, I rewrote FastQC (the existing validation tool widely used by the bioinformatics community) in Rust in a week, and it turned out to be 5x faster than the existing tool.

I'm a science enthusiast, an avid programmer, and a quick learner. I love learning new stuff, I enjoy solving problems (either by my own, or collaboratively with other people), and I especially like creating applications or features that save time for a lot of people. I also love to share my knowledge to others, which explains my contributions to Mozilla's open-source software. It's all these experiences that have built my confidence over time, and that's also why I think I'm fit for this job.

I don't really know whether this is enough to convince you about my potential for this position, but I can assure you that I'll prove my worth if I'm given a chance. I would enjoy discussing this with you in the weeks to come. In the meantime, I'm enclosing my curriculum vitae along with this letter.

Thank you very much for taking your time for consideration.

$\ $

Yours sincerely,
\\
Ravi Shankar

% \clearpage\end{CJK*}                             % if you are typesetting your resume in Chinese using CJK; the \clearpage is required for fancyhdr to work correctly with CJK, though it kills the page numbering by making \lastpage undefined
\end{document}
