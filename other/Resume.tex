% Copyright 2006-2015 Xavier Danaux (xdanaux@gmail.com).
%
% This work may be distributed and/or modified under the
% conditions of the LaTeX Project Public License version 1.3c,
% available at http://www.latex-project.org/lppl/.

\documentclass[11pt,a4paper,sans]{moderncv}        % possible options include font size ('10pt', '11pt' and '12pt'), paper size ('a4paper', 'letterpaper', 'a5paper', 'legalpaper', 'executivepaper' and 'landscape') and font family ('sans' and 'roman')

% moderncv themes
\moderncvstyle{banking}                            % style options are 'casual' (default), 'classic', 'banking', 'oldstyle' and 'fancy'
\moderncvcolor{black}                              % color options 'black', 'blue' (default), 'green', 'grey', 'orange', 'purple' and 'red'
% \renewcommand{\familydefault}{\sfdefault}          % to set the default font; use '\sfdefault' for the default sans serif font, '\rmdefault' for the default roman one, or any tex font name
\nopagenumbers                                     % uncomment to suppress automatic page numbering for CVs longer than one page

% character encoding
% \usepackage[utf8]{inputenc}                        % if you are not using xelatex ou lualatex, replace by the encoding you are using
% \usepackage{CJKutf8}                               % if you need to use CJK to typeset your resume in Chinese, Japanese or Korean

% adjust the page margins
\usepackage[scale=0.75]{geometry}
% \setlength{\hintscolumnwidth}{3cm}                 % if you want to change the width of the column with the dates
% \setlength{\makecvtitlenamewidth}{10cm}            % for the 'classic' style, if you want to force the width allocated to your name and avoid line breaks. be careful though, the length is normally calculated to avoid any overlap with your personal info; use this at your own typographical risks...

% personal data
\name{Ravi}{Shankar}
% \title{Resumé}
% \address{street and number}{postcode city}{country} % the "postcode city" and "country" arguments can be omitted or provided empty
\address{\textbf{UG Final Year (B.E. Aeronautics)}}{Madras Institute of Technology}
\phone[mobile]{+91~9551404646}                     % the optional "type" of the phone can be "mobile" (default), "fixed" or "fax"
% \phone[fixed]{+2~(345)~678~901}
% \phone[fax]{+3~(456)~789~012}
\email{wafflespeanut@gmail.com}
% \homepage{wafflespeanut.github.io}
% \social[linkedin]{wafflespeanut}
% \social[twitter]{wafflespeanut}
\social[github]{wafflespeanut}
% \extrainfo{}
% \photo[64pt][0.4pt]{picture}                       % '64pt' is the height the picture must be resized to, 0.4pt is the thickness of the frame around it (put it to 0pt for no frame) and 'picture' is the name of the picture file
% \quote{Some quote}                                 
% bibliography adjustements (only useful if you make citations in your resume, or print a list of publications using BibTeX)
%   to show numerical labels in the bibliography (default is to show no labels)
\makeatletter\renewcommand*{\bibliographyitemlabel}{\@biblabel{\arabic{enumiv}}}\makeatother
%   to redefine the bibliography heading string ("Publications")
%\renewcommand{\refname}{Articles}

% bibliography with mutiple entries
%\usepackage{multibib}
%\newcites{book,misc}{{Books},{Others}}

\definecolor{linky}{rgb}{0.1, 0.2, 0.9}
\newcommand\chref[3][linky]{\href{#2}{\color{#1}#3}}

%----------------------------------------------------------------------------------
%            content
%----------------------------------------------------------------------------------

\begin{document}
% \begin{CJK*}{UTF8}{gbsn}                         % to typeset your resume in Chinese using CJK

%-----       resume       ---------------------------------------------------------

\vspace*{-2\baselineskip}
\makecvtitle
\vspace{-2\baselineskip}

%\section{Education}
%\cventry{year--year}{Degree}{Institution}{City}{\textit{Grade}}{Description}  % arguments 3 to 6 can be left empty
%\cventry{year--year}{Degree}{Institution}{City}{\textit{Grade}}{Description}

% \section{Master thesis}
% \cvitem{title}{\emph{Title}}
% \cvitem{supervisors}{Supervisors}
% \cvitem{description}{Short thesis abstract}

\section{Projects}
\cventry{Madras Institute of Technology}{Prof. Jayaraman}{Aircraft Design Project}{December, 2014 -- 2015}{}
{\begin{itemize}
\item Studied and calculated the various parameters required for designing a 420-seater "jumbo jet" aircraft.
\item \chref{https://github.com/Wafflespeanut/scripts/tree/5610248de2d47311f128fecd015e2af8becca26f/python/Course}{Wrote a number of Python scripts} for automating the data collection and plotting, which reduced a great deal of time for the fellow undergrads.
\end{itemize}}
\cventry{Genome Life Sciences}{Giriraj Namachivayam (Lead Developer)}{Backend Developer Intern}{January, 2015 -- Present}{}
{\begin{itemize}
\item Wrote utilities to parse and analyze large quantities of chromosome data.
\item Introduced the Rust language to the team, and rewrote a number of Bash and Python scripts in Rust, which showed a drastic improvement in performance.
\item Currently writing a parser in Rust to process data, which is expected to bring down the processing time from seconds to a few milliseconds.
\end{itemize}}
\cventry{Madras Institute of Technology}{Prof. Arumugam}{Improving the fracture toughness of tapered composites}{January, 2015 -- Present}{}
{\begin{itemize}
\item Conducted various tensile tests and acoustic experiments on tapered composites and studied about the discontinuous stress distributions in each lamina.
\item Currently trying to improve the fracture toughness by the addition of filler material in an iterative process.
\end{itemize}}

\section{Open source contributions}
\subsection{Mozilla}
\begin{itemize}
\item \chref{https://github.com/servo/servo/commits?author=Wafflespeanut}{Contributor} and \chref{https://blog.servo.org/2016/01/11/twis-47/}{reviewer} for the \chref{https://github.com/servo/servo}{Servo browser engine} project over the last few months, primarily concentrating on the python code used by the build system and mentoring the newcomers.
\item Occassional contributor to the \chref{https://github.com/rust-lang/rust}{Rust programming language}, its documentation and related tooling.
\item \textbf{\chref{https://mozillians.org/en-US/u/wafflespeanut}{Mozillian}} since the summer of 2015.
\end{itemize}
\subsection{Personal projects}
\begin{itemize}
\item \textbf{\chref{https://github.com/Wafflespeanut/biographer}{Biographer}}: A command-line based private diary written in Python, which allows users to write their everyday stories, view them, or search through them later. It makes use of a simple shifting cipher to encrypt/decrypt the contents. It also contains a Rust library, which uses FFI and concurrency to reduce the searching time by a factor of $\approx 100$.
\item \textbf{\chref{https://github.com/Wafflespeanut/free-fall}{Free fall}}: A terminal based ASCII 2D game written in Rust, where the users try to save a jumper from hitting the cliffs. The game makes use of the terminal's raw mode and interacts with the Unix C libraries for polling the keystroke inputs and prints thousands of characters frame by frame to indicate motion.
\item \textbf{\chref{https://github.com/Wafflespeanut/flight-2016}{Flight '16}}: A \chref{http://wafflespeanut.github.io/flight-2016}{responsive website} written in pure HTML/JS/CSS (for our dept. symposium) without the use of any external libraries. Since most of the audience were 2G users, it's optimized in such a way that the desktop version consumes atmost 5 MB, and the mobile version consumes barely 1.5 MB, which brings the loading time to a few hundred milliseconds.
\item \textbf{\chref{https://github.com/joelewis/carrot}{Carrot}}: An MVC-based webapp which lets organizations to notify their users of the changes made to their webapp(s) by using a simple \texttt{<script>} tag embedded on their webpage, which gets data from our server through the JSONP technique.
\end{itemize}

\newpage
% \section{Languages}
% \cvitemwithcomment{Language 1}{Skill level}{Comment}

% \section{Skills:}
% \cvdoubleitem{category 1}{XXX, YYY, ZZZ}{category 2}{XXX, YYY, ZZZ}

\section{Programming skills}
\cvitem{Languages}{Python, Rust, HTML5, Javascript, CSS, Bash, and some \textrm{\LaTeX}}
\cvitem{Technologies}{Git, Mathematica, some Django and Angular JS}

\section{Key Courses Undertaken}
\cvitem{Aeronautics}{Aircraft Structures, Aerodynamics, Propulsion, Flight Mechanics, Aircraft Stability}
\cvitem{Mathematics}{Numerical Methods, Transform Techniques \& PDE, Finite Element Method}

\section{Public speaking}
\begin{itemize}
\item Conducted introductory hands-on sessions for Python in college
\item Volunteer at ``Mozboot'' sessions (conducted by Mozilla) in college
\end{itemize}

\section{Miscellaneous}
\begin{itemize}
\item Blogger for the past two years on \chref{https://wafflescrazypeanut.wordpress.com/}{wafflescrazypeanut.wordpress.com} and \chref{https://wafflespeanut.github.io/}{wafflespeanut.github.io}
\item I also play the Indian flute, and sometimes juggle
\end{itemize}

% \section{Extra 1}
% \cvlistitem{Item 1}
% \cvlistitem{Item 2. This item is particularly long and therefore normally spans over several lines. Did you notice the indentation when the line wraps?}

% \section{Extra 2}
% \cvlistdoubleitem{Item 1}{Item 4}
% \cvlistdoubleitem{Item 2}{Item 5\cite{book1}}
% \cvlistdoubleitem{Item 3}{Item 6. Like item 3 in the single column list before, this item is particularly long to wrap over several lines.}

% \section{References}
% \begin{cvcolumns}
%   \cvcolumn{Category 1}{\begin{itemize}\item Person 1\item Person 2\item Person 3\end{itemize}}
%   \cvcolumn{Category 2}{Amongst others:\begin{itemize}\item Person 1, and\item Person 2\end{itemize}(more upon request)}
%   \cvcolumn[0.5]{All the rest \& some more}{\textit{That} person, and \textbf{those} also (all available upon request).}
% \end{cvcolumns}

% Publications from a BibTeX file without multibib
%  for numerical labels: \renewcommand{\bibliographyitemlabel}{\@biblabel{\arabic{enumiv}}}% CONSIDER MERGING WITH PREAMBLE PART
%  to redefine the heading string ("Publications"): \renewcommand{\refname}{Articles}
% \nocite{*}
% \bibliographystyle{plain}
% \bibliography{publications}                      % 'publications' is the name of a BibTeX file

% Publications from a BibTeX file using the multibib package
% \section{Publications}
% \nocitebook{book1,book2}
% \bibliographystylebook{plain}
% \bibliographybook{publications}                  % 'publications' is the name of a BibTeX file
% \nocitemisc{misc1,misc2,misc3}
% \bibliographystylemisc{plain}
% \bibliographymisc{publications}                  % 'publications' is the name of a BibTeX file

\clearpage

%-----       letter       ---------------------------------------------------------

% recipient data
% \recipient{[Company] Recruitment team}{[Company]}
% \date{}
% \opening{Hello,}

% \closing{Yours faithfully,}

% \enclosure[Attached]{curriculum vit\ae{}}        % use an optional argument to use a string other than "Enclosure", or redefine \enclname

% \makelettertitle

% \clearpage\end{CJK*}                             % if you are typesetting your resume in Chinese using CJK; the \clearpage is required for fancyhdr to work correctly with CJK, though it kills the page numbering by making \lastpage undefined
\end{document}
