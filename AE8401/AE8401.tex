\documentclass{article}
\title{\Huge Aircraft Structures - I}
\Large{\author{$\mathcal{WAFFLE'S\ \ CRAZY\ \ PEANUT}$}}
\date{(Last updated: 21/12/13)}
\usepackage{enumerate}
\usepackage{amssymb}
\usepackage{tikz-qtree}
\usetikzlibrary{trees}
\usepackage{graphicx}
\usepackage{caption,subcaption}
\usepackage{calc}
\usepackage{amsmath}
\newcommand{\para}[1]{\paragraph{#1}\mbox{}\\}
\usepackage{pgf,tikz}
\usetikzlibrary{arrows}
\usetikzlibrary{decorations.markings}
\pagestyle{empty}
\pagenumbering{gobble}
\newcommand{\degre}{\ensuremath{^\circ}}
\newcommand{\twoheaduparrow}{\mathrel{\rotatebox{90}{$\twoheadrightarrow$}}}
\begin{document}
\maketitle
\textbf{\section{\huge Static Determinacy of Structures}}
\
{\Large 
\para{\LARGE Mechanics Recall:}
\newline
\textbf{Coplanar forces} - confined to a common plane, can be parallel or concurrent. Their general resolution is,
{\LARGE $$\sum F_x=0,\ \sum F_y=0,\ \sum M_P=0$$}
\\
Sum of forces along the ``mutually perpendicular" directions is zero, and the sum of moments (including those contributed by the forces) about any point is \textbf{zero}.
\begin{enumerate}[(i)]
\item For \textit{concurrent forces}, the equilibrium conditions $\sum F_x=0$ and $\sum F_y=0$ are enough.
\item For \textit{parallel forces} the conditions $\sum F=0$ along direction of the forces, and $\sum M_P=0$ are suffice to resolve them.
\end{enumerate}
\newpage
\subsection{\LARGE Plane Truss Analysis}
$\ $
\paragraph{\Large Static Indeterminacy:}
\begin{itemize}
\item If the number of unknown reactions developed $>$ number of static equilibrium equations, then it is a  \textbf{statically indeterminate} structure.

(While truss works can make such cases statically determinate ``externally", the trusses themselves need to be statically determinate ``internally").
\item The stability of a truss work is governed by
{\LARGE $$n=2j-3$$} $n=$ number of members, $\ j=$ number of joints
\begin{enumerate}[(a)]
\item $\mathrm{LHS=RHS}$: Structure is statically determinate, and it's a fail-safe design (i.e) it can survive on the event of failure (other members can still be able to hold the structure).
\item $\mathrm{LHS>RHS}$: Structure is statically indeterminate ``internally", and it corresponds to a safe-life design (i.e.) they can live a long time without any repairing requirements.
\item $\mathrm{LHS<RHS}$: It's no longer a structure. It's a mechanism, which collapses as a whole.
\end{enumerate}
\end{itemize}
\newpage
\paragraph{\Large Assumptions made:}
\begin{itemize}
\item Members are pin-joined.
\item Members resist the point loads in the form of tension and compression.
\item Loads are applied only at joints.
\end{itemize}
\begin{flushleft}
\definecolor{qqccqq}{rgb}{0,0.8,0}
\definecolor{ffqqqq}{rgb}{1,0,0}
\definecolor{zzttqq}{rgb}{0.6,0.2,0}
\definecolor{qqqqcc}{rgb}{0,0,0.8}
\begin{tikzpicture}[scale=1.2]\hspace*{-5.5cm}
\clip(-18.6,-3.54) rectangle (3.89,7.74);
\draw [shift={(-13,1)},color=qqqqcc,fill=qqqqcc,fill opacity=0.1] (0,0) -- (0:1.09) arc (0:45:1.09) -- cycle;
\fill[color=zzttqq,fill=zzttqq,fill opacity=0.1] (-13,1) -- (-13.5,0) -- (-12.5,0) -- cycle;
\fill[color=zzttqq,fill=zzttqq,fill opacity=0.1] (-3,1) -- (-3.5,0) -- (-2.5,0) -- cycle;
\draw (-3,1)-- (-13,1);
\draw (-13,1)-- (-8,6);
\draw (-8,6)-- (-3,1);
\draw (-8,1)-- (-5.5,3.5);
\draw (-8,1)-- (-10.5,3.5);
\draw (-8,6)-- (-8,1);
\draw [color=zzttqq] (-13,1)-- (-13.5,0);
\draw [color=zzttqq] (-13.5,0)-- (-12.5,0);
\draw [color=zzttqq] (-12.5,0)-- (-13,1);
\draw [color=zzttqq] (-3,1)-- (-3.5,0);
\draw [color=zzttqq] (-3.5,0)-- (-2.5,0);
\draw [color=zzttqq] (-2.5,0)-- (-3,1);
\draw(-3.39,-0.22) circle (0.2cm);
\draw(-2.96,-0.22) circle (0.2cm);
\draw(-2.53,-0.22) circle (0.2cm);
\draw (-13.44,-0.32)-- (-13.19,-0.01);
\draw (-13.64,-0.31)-- (-13.39,0.01);
\draw (-13.18,-0.31)-- (-12.93,0.01);
\draw (-12.98,-0.31)-- (-12.73,0.01);
\draw (-12.76,-0.31)-- (-12.52,0.01);
\draw (-13.54,1.75) node[anchor=north west] {$\mathrm A$};
\draw (-3.05,1.65) node[anchor=north west] {$\mathrm E$};
\draw (-8.3,6.75) node[anchor=north west] {$\mathrm C$};
\draw (-11.13,4.28) node[anchor=north west] {$\mathrm B$};
\draw (-5.35,4.3) node[anchor=north west] {$\mathrm D$};
\draw (-8.17,0.94) node[anchor=north west] {$\mathrm F$};
\draw[->,arrow head=0.15, line width=1.5pt, color=ffqqqq] (-5.5,5.5) -- (-5.5,3.7);
\draw[->, arrow head=0.15, line width=1.5pt, color=ffqqqq] (-10.5,5.5) -- (-10.5,3.7);
\draw[->, arrow head=0.15, line width=1.5pt, color=ffqqqq] (-12,3.5) -- (-10.7,3.5);
\draw[->, arrow head=0.15, line width=1.5pt, color=ffqqqq] (-5.3,3.5) -- (-4,3.52);
\draw (-10.85,6.31) node[anchor=north west] {$1\ \mathrm {kN}$};
\draw (-13.09,4.07) node[anchor=north west] {$2\ \mathrm {kN}$};
\draw (-3.94,4.05) node[anchor=north west] {$3\ \mathrm {kN}$};
\draw (-5.84,6.29) node[anchor=north west] {$4\ \mathrm {kN}$};
\draw [->,dash pattern=on 3pt off 3pt] (-8,-2) -- (-3,-2);
\draw [->, dash pattern=on 3pt off 3pt] (-8,-2) -- (-13,-2);
\draw (-8.6,-1.24) node[anchor=north west] {$\mathrm {10\ m}$};
\draw [->, arrow head=0.15in, line width=1.5pt, color=ffqqqq] (-3,-1.73) -- (-3,-0.53);
\draw [->, arrow head=0.15in, line width=1.5pt, color=ffqqqq] (-13.03,-1.73) -- (-13.01,-0.48);
\draw (-12.95,-0.64) node[anchor=north west] {$\vec V_A=0.5\ \mathrm{kN}$};
\draw [->, arrow head=0.15in, line width=1.5pt, color=ffqqqq] (-14.69,1.01) -- (-13.2,0.99);
\draw (-16.3,1.1) node[anchor=north west] {$\vec H_A=-5\ \mathrm{kN}$};
\draw (-5.45,-0.56) node[anchor=north west] {$\vec V_E=4.5\ \mathrm{kN}$};
\draw [line width=2pt,dash pattern=on 1pt off 3pt on 5pt off 4pt,color=qqccqq] (-9.02,6.5)-- (-9.02,-0.5);
\draw [line width=2pt,dash pattern=on 1pt off 3pt on 5pt off 4pt,color=qqccqq] (-7.01,6.5)-- (-7.01,-0.5);
\draw [line width=2pt,dash pattern=on 1pt off 3pt on 5pt off 4pt,color=qqccqq] (-3.6,2.93)-- (-6.05,0.47);
\draw (-9.35,7.1) node[anchor=north west] {$a$};
\draw (-9.35,-0.45) node[anchor=north west] {$a'$};
\draw (-7.25,7.2) node[anchor=north west] {$b$};
\draw (-7.22,-0.48) node[anchor=north west] {$b'$};
\draw (-3.65,3.4) node[anchor=north west] {$c$};
\draw (-6.6,0.51) node[anchor=north west] {$c'$};
\draw [->, arrow head=0.1in, line width=1pt] (-13,1) -- (-11.73,1);
\draw [->, arrow head=0.1in, line width=1pt] (-13,1) -- (-12.06,1.94);
\draw [->, arrow head=0.1in, line width=1pt] (-10.5,3.5) -- (-11.32,2.68);
\draw [->, arrow head=0.1in, line width=1pt] (-10.5,3.5) -- (-9.78,2.78);
\draw [->, arrow head=0.1in, line width=1pt] (-10.5,3.5) -- (-9.677,4.333);
\draw [->, arrow head=0.1in, line width=1pt] (-8,6) -- (-8.74,5.26);
\draw [->, arrow head=0.1in, line width=1pt] (-8,6) -- (-8,4.89);
\draw [->, arrow head=0.1in, line width=1pt] (-8,6) -- (-7.27,5.27);
\draw [->, arrow head=0.1in, line width=1pt] (-5.5,3.5) -- (-6.27,2.73);
\draw [->, arrow head=0.1in, line width=1pt] (-5.5,3.5) -- (-6.24,4.24);
\draw [->, arrow head=0.1in, line width=1pt] (-5.5,3.5) -- (-4.71,2.71);
\draw [->, arrow head=0.1in, line width=1pt] (-3,1) -- (-4.21,1);
\draw [->, arrow head=0.1in, line width=1pt] (-3,1) -- (-3.81,1.81);
\draw [->, arrow head=0.1in, line width=1pt] (-8,1) -- (-7.13,1);
\draw [->, arrow head=0.1in, line width=1pt] (-8,1) -- (-8.87,1);
\draw [->, arrow head=0.1in, line width=1pt] (-8,1) -- (-8,1.99);
\draw [->, arrow head=0.1in, line width=1pt] (-8,1) -- (-7.36,1.64);
\draw [->, arrow head=0.1in, line width=1pt] (-8,1) -- (-8.64,1.64);
\draw (-12.59,3.22) node[anchor=north west] {$\vec F_{AB}$};
\draw (-11.6,1.03) node[anchor=north west] {$\vec F_{AF}$};
\draw (-10.31,5.5) node[anchor=north west] {$\vec F_{BC}$};
\draw (-6.79,5.6) node[anchor=north west] {$\vec F_{CD}$};
\draw (-3.86,2.57) node[anchor=north west] {$\vec F_{DE}$};
\draw (-4.93,1.02) node[anchor=north west] {$\vec F_{EF}$};
\draw (-6.3,2.9) node[anchor=north west] {$\vec F_{DF}$};
\draw (-8.06,4.35) node[anchor=north west] {$\vec F_{CF}$};
\draw (-10.45,2.8) node[anchor=north west] {$\vec F_{BF}$};
\draw [->,dash pattern=on 3pt off 3pt] (-15,3.01) -- (-15,6.01);
\draw [->, dash pattern=on 3pt off 3pt] (-15,3.01) -- (-15,1.01);
\draw (-14.85,4.01) node[anchor=north west] {$\mathrm {5\ m}$};
\begin{scriptsize}
\draw[color=qqqqcc] (-12.3,1.32) node {$45\textrm{\degre}$};
\end{scriptsize}
\end{tikzpicture}
\end{flushleft}
\begin{center}
\textbf{{\Large Figure 1: Plane truss}}
\end{center}
\paragraph{\Large Methods:}
\begin{enumerate}[(i)]
\item Method of Joints
\item Method of Section
\item Method of Shear
\end{enumerate}
\subsection{\LARGE Method of Joints}
Resolve the forces joint by joint.
\paragraph{\Large Procedure:}
\begin{enumerate}[(a)]
\item Find the reaction at the supports, and check whether the truss work is statically determinate externally and internally.
\item Choose a ``resolvable" joint* (where the number of unknown reactions $\leq$ the number of equilibrium equations) and find the tension on the members.
\item As the forces are concurrent, the equilibrium equations $F_x=0$ and $F_y=0$ are suffice to resolve.
\newline

*\textbf{Note:} Some joints may have a greater number of unknowns. For e.g., joint $\mathrm F$ in the above truss has five tensile forces acting on it. They can be used to check your calculations.
\end{enumerate}
\newpage
\paragraph{\Large Keep in mind:}
\begin{enumerate}[1)]
\item Forces are \textit{assumed} to be tensile at all joints (i.e) they act away from the joints.
\item The mutually perpendicular directions can be in any orientation. Even a tangent and normal to a given joint can do the resolution of forces.
\item Don't confuse yourselves with \textbf{Newton's third law}. This convention does not mean that.

For e.g., in the above truss, $F_{AB}$ is the tension between $\mathrm A$ and $\mathrm B$. At joint $\mathrm A$, $\vec F_{AB}$ acts from $\mathrm A\to\mathrm B$, whereas at the joint $\mathrm B$, it acts from $\mathrm B\to\mathrm A$, according to our assumption (see \textbf{1st point}).

But, it's true. Because, tension is similar to reaction. You resolve it once, and get the direction, then that direction remains the same throughout the problem. So, $\vec F_{AB}=\vec F_{BA}$
\end{enumerate}
\subsection{\LARGE Method of Section:}
Slice the structure into sections. As the truss needs tensile and compressive forces to be stable, you apply the tensile forces manually at the loose ends of the section, and resolve them.
\newpage
$\\ $
(It's very similar to D'Alembert's principle, used to analyze an accelerating object, wherein you provide an opposing force to balance the acceleration of the object, and thereby resolve it using static equilibrium equations. This analysis is common in classical mechanics)
$\\ $

Slicing is shown in Fig.1 by $aa'$, $bb'$ and $cc'$.

\paragraph{\Large Procedure:}
\begin{enumerate}[(a)]
\item Slice the structure appropriately between joints. No more than 3 unknown forces should appear, for the sliced part to be statically determinate ``internally".
\item Use the moment equilibrium condition (i.e.) moment about any point, $\vec M_P=0$.
\newline

\textbf{\Large Keep in mind:} Moment is simply force times the ``perpendicular distance" $(\vec M=\vec r\times \vec F)$. So, the forces that pass through the point do not contribute any moment. And, while resolving, stick to a particular direction (clockwise/anticlockwise).
\newline

For e.g., taking left slice of $aa'$ in Fig.1, moments $\vec M_A$, $\vec M_F$ and $\vec M_B$ can be equated to zero. Regarding the perpendicular direction, (for e.g.) while using $\vec M_F=0$, the force $\mathrm{1\ kN}$ is at $\mathrm{r=2.5\ m}$ (horizontal), whereas $\mathrm{2\ kN}$ is at $\mathrm{r=2.5\ m}$ (vertical).
\end{enumerate}
\newpage
\subsection{\LARGE Space Truss}
Analyze the forces on members from the given orthographically projected truss work.
\para{\Large Moment recall:}
\\
Moment is always taken about a line. So, the moment of a force about a line parallel to its direction is \textbf{zero!}
$$\mathrm{Moment=Force\times Perpendicular\ distance}$$
Moment is a vector. Its direction is perpendicular to the plane containing $\vec r$ and $\vec F$
{\LARGE $$\vec M=\vec r \times \vec F$$}
Resolution of moments, and magnitude of net moment are similar to that of the forces,
{\LARGE $$M=\sqrt{{M_x}^2+{M_y}^2+{M_z}^2}$$}
\paragraph{\Large Note:}
\begin{itemize}
\item When a force is transferred from one point to another point, it goes as a force with same magnitude and direction, but also with a moment.
\item Just as forces are indicated by arrows ($\rightarrow $), moments are indicated by double-headed arrows ({\huge $\twoheadrightarrow $}). Their rotation is given by one of the thumb rules, left/right depends on the question.
\end{itemize}
\newpage
\paragraph{\Large Procedure:}
\begin{enumerate}[(a)]
\item If the given members in 3D space, then project them orthographically in 2D (front and side views).
\item Tabulate the distance of the members relative to the vertical or horizontal (depending on {\LARGE $D$},{\LARGE $S$}, or {\LARGE $V$}).
\item Find the magnitude of the net distance using,
{\LARGE $$|L|=\sqrt{D^2+S^2+V^2}$$}
\item Find {\LARGE $D\over L$},{\LARGE $S\over L$}, and {\LARGE $V\over L$}.
\item Now that the direction cosines are found, apply the force equilibrium to find the forces on members,
\begin{center}
{\LARGE $\sum D=0$}, {\LARGE $\sum S=0$}, and {\LARGE $\sum V=0$}
\end{center}
\end{enumerate}
$\ $
\\
\textbf{\Large Note:} 
\newpage
\para{\Large Simply-supported beams: (Formulas recall)}
\begin{itemize}
\item For a point load $P$ at a distance {\LARGE $a$} from the support,
\end{itemize}
\definecolor{qqqqcc}{rgb}{0,0,0.8}
\definecolor{qqqqff}{rgb}{0,0,1}
\definecolor{ffqqqq}{rgb}{1,0,0}
\definecolor{zzttqq}{rgb}{0.6,0.2,0}
\begin{tikzpicture}[scale=0.8,line cap=round,line join=round,>=triangle 45]
\hspace*{2cm}\raisebox{1.5pt}
\clip(-3.72,-3.18) rectangle (19.18,8.52);
\fill[color=zzttqq,fill=zzttqq,fill opacity=0.1] (2.05,3.69) -- (1.55,2.69) -- (2.55,2.69) -- cycle;
\fill[color=zzttqq,fill=zzttqq,fill opacity=0.1] (12.05,3.69) -- (11.55,2.69) -- (12.55,2.69) -- cycle;
\draw [color=zzttqq] (2.05,3.69)-- (1.55,2.69);
\draw [color=zzttqq] (1.55,2.69)-- (2.55,2.69);
\draw [color=zzttqq] (2.55,2.69)-- (2.05,3.69);
\draw [color=zzttqq] (12.05,3.69)-- (11.55,2.69);
\draw [color=zzttqq] (11.55,2.69)-- (12.55,2.69);
\draw [color=zzttqq] (12.55,2.69)-- (12.05,3.69);
\draw(11.66,2.47) circle (0.2cm);
\draw(12.09,2.47) circle (0.2cm);
\draw(12.52,2.47) circle (0.2cm);
\draw (1.61,2.37)-- (1.85,2.68);
\draw (1.41,2.38)-- (1.66,2.7);
\draw (1.87,2.38)-- (2.12,2.7);
\draw (2.07,2.38)-- (2.32,2.7);
\draw (2.28,2.38)-- (2.53,2.7);
\draw [->,dash pattern=on 2pt off 2pt] (3.86,0.69) -- (5.46,0.69);
\draw [->,dash pattern=on 2pt off 2pt] (3.86,0.69) -- (2.05,0.69);
\draw [->,color=ffqqqq] (12.05,0.96) -- (12.05,2.16);
\draw [->,color=ffqqqq] (2.02,0.96) -- (2.04,2.21);
\draw [line width=1.2pt] (2.05,3.69)-- (12.05,3.69);
\draw (1,4.4) node[anchor=north west] {$\mathrm A$};
\draw (12.2,4.4) node[anchor=north west] {$\mathrm B$};
\draw (1,4.4) node[anchor=north west] {$\mathrm A$};
\draw (5.6,4.55) node[anchor=north west] {$\mathrm C$};
\draw [->,color=qqqqcc] (5.44,5.4) -- (5.46,3.86);
\draw (4.8,6.2) node[anchor=north west] {$P\ $};
\draw [->,dash pattern=on 2pt off 2pt] (8.86,0.69) -- (12.05,0.69);
\draw [->,dash pattern=on 2pt off 2pt] (8.86,0.69) -- (5.46,0.69);
\draw (3.6,1.4) node[anchor=north west] {$a$};
\draw (8.6,1.55) node[anchor=north west] {$b$};
\draw (0.9,1.5) node[anchor=north west] {$V_A$};
\draw (12.1,1.5) node[anchor=north west] {$V_B$};
\begin{scriptsize}
\draw [color=qqqqff] (5.44,3.69)-- ++(-3.0pt,-3.0pt) -- ++(6.0pt,6.0pt) ++(-6.0pt,0) -- ++(6.0pt,-6.0pt);
\end{scriptsize}
\end{tikzpicture}
\\
\begin{enumerate}[(i)]
\item Shear force distribution,
{\LARGE $$\frac{Pb}{a+b}-P$$}
\item Bending moment distribution,
{\LARGE $$\frac{Pb}{a+b}-P(x-a)$$}
\item Deflection at point C,
\LARGE{$$\delta_c=\frac{P}{6EI}\bigg(\frac{bx^3}{a+b}-\frac{abx}{a+b}(a+2b)-(x-a)^3\bigg)$$}
\end{enumerate}
Simplifying for {\LARGE $a=b=x=\frac{L}2$},
{\LARGE $$\delta_c=\frac{-PL^3}{48EI}$$}
\end{document}